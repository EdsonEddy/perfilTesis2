\begin{figure}[!h]
\centering
\begin{tikzpicture}
\matrix (M) [matrix of nodes,
column sep=0pt,
row sep=0pt,
nodes={draw,fill=gray!50,minimum width=.5cm,outer sep=0pt,minimum
height=.5cm,anchor=center},
column 1/.style={minimum height=.8cm}]{
\mbox{} &[2mm] 2 & \mbox{} &[2mm] 5 & - &[2mm] & &[2mm] & \\
\mbox{} & 1 & \mbox{} & 5 & \mbox{} & 3 & \mbox{} & 4 & - \\
\mbox{} & 2 & \mbox{} & 4 & - & & & & \\
\mbox{} & 2 & \mbox{} & 5 & \mbox{} & 3 & - & & \\
\mbox{} & 4 & \mbox{} & 1 & \mbox{} & 2 & - & & \\
};
\foreach \i in {1,2,3,4,5}{
\path (M-\i-1) [late options={label=left:\i}];
\draw[->] (M-\i-1.center)--(M-\i-2.west);
\draw[->] (M-\i-3.center)--(M-\i-4.west);
}
\draw[->] (M-2-5.center)--(M-2-6.west);
\draw[->] (M-4-5.center)--(M-4-6.west);
\draw[->] (M-5-5.center)--(M-5-6.west);
\draw[->] (M-2-7.center)--(M-2-8.west);
\end{tikzpicture}
\caption{Ejemplo de la lista de adyacencia de un grafo}
Para un grafo $G=(V,E)$, donde $V = \{(1,2),(1,5),(2,1),(2,5),(2,3),(2,4),(3,2),(3,4),(4,2),(4,5),(4,3),(5,4),(5,1),(5,2)\}$. \\Fuente: \cite{Cormen2009}.
\label{listaAdy}
\end{figure}
