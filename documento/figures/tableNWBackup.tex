\begin{figure}[!h]
\centering
\begin{tikzpicture}[
table/.style={
    matrix of nodes,
    row sep=-\pgflinewidth,
    column sep=-\pgflinewidth,
    nodes={rectangle,text width=1.5em,align=center,draw},
    text depth=1.2ex,
    text height=2ex,
    nodes in empty cells,
}
]
        \matrix (mat) [table]{
        &\_&A&G&C&A&T&G&C\\
        \_&0&-1&-2&-3&-4&-5&-6&-7\\
        A&-1&2&1&0&-1&-2&-3&-4\\
        C&-2&1&1&3&2&1&0&-1\\
        A&-3&0&0&2&5&4&3&2\\
        A&-4&-1&-1&1&4&4&3&2\\
        T&-5&-2&-2&0&3&6&5&4\\
        C&-6&-3&-3&0&2&5&5&7\\
        C&-7&-4&-4&-1&1&4&4&7\\
    };
    \fill[black,opacity=0.2] (mat-2-2.north west) rectangle (mat-2-2.south east);
    \fill[black,opacity=0.2] (mat-3-3.north west) rectangle (mat-3-3.south east);
    \fill[black,opacity=0.2] (mat-3-4.north west) rectangle (mat-3-4.south east);
    \fill[black,opacity=0.2] (mat-4-5.north west) rectangle (mat-4-5.south east);
    \fill[black,opacity=0.2] (mat-5-6.north west) rectangle (mat-5-6.south east);
    \fill[black,opacity=0.2] (mat-6-6.north west) rectangle (mat-6-6.south east);
    \fill[black,opacity=0.2] (mat-7-7.north west) rectangle (mat-7-7.south east);
    \fill[black,opacity=0.2] (mat-8-8.north west) rectangle (mat-8-8.south east);
    \fill[black,opacity=0.2] (mat-9-9.north west) rectangle (mat-9-9.south east);
    \foreach \x in {1,2,3,4,5,6,7,8,9}
    {
        \draw
        ([xshift=-.125\pgflinewidth]mat-\x-1.north west) --
        ([xshift=-.125\pgflinewidth]mat-\x-9.north east);
    }
    \draw
        ([xshift=-.125\pgflinewidth]mat-9-1.south west) --
        ([xshift=-.125\pgflinewidth]mat-9-9.south east);
    \foreach \y in {1,2,3,4,5,6,7,8,9}
    {
        \draw
        ([yshift=.5\pgflinewidth]mat-1-\y.north east) --
        ([yshift=.5\pgflinewidth]mat-9-\y.south east);
    }
    \draw
        ([xshift=-.125\pgflinewidth]mat-1-1.north west) --
        ([xshift=-.125\pgflinewidth]mat-9-1.south west);
    \begin{scope}[shorten >=7pt,shorten <= 7pt]
    \draw[->,line width=1pt]  (mat-9-9.center) -- (mat-8-8.center);
    \draw[->,line width=1pt]  (mat-8-8.center) -- (mat-7-7.center);
    \draw[->,line width=1pt]  (mat-7-7.center) -- (mat-6-6.center);
    \draw[->,line width=1pt]  (mat-6-6.center) -- (mat-5-6.center);
    \draw[->,line width=1pt]  (mat-5-6.center) -- (mat-4-5.center);
    \draw[->,line width=1pt]  (mat-4-5.center) -- (mat-3-4.center);
    \draw[->,line width=1pt]  (mat-3-4.center) -- (mat-3-3.center);
    \draw[->,line width=1pt]  (mat-3-3.center) -- (mat-2-2.center);
    \end{scope}
\end{tikzpicture}
\caption{Ejemplo del algoritmo de Needleman-Wunsch}
Ejemplo: A=``ACAATCC'' y B=``AGCATGC'', la puntuación de la alineación = 7. \\Fuente: \cite{Halim2019}.
\label{tableNW}
\end{figure}
