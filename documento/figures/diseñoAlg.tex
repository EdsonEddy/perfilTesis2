\begin{figure}[!h]
\centering
\begin{tikzpicture}[scale=0.9,
algorithm/.style ={rectangle, minimum width=3cm, minimum height=1cm,text centered, text width=3cm, draw=black},
border/.style ={rectangle, minimum width=2.5cm, minimum height=1cm,text centered, text width=2.75cm, draw=black},
module/.style={rectangle, minimum width=3cm, minimum height=1cm, text centered, text width=5cm, draw=black},
io/.style ={trapezium, trapezium left angle=70, trapezium right angle=110, minimum height=1cm,text centered, text width=2.5cm, draw=black},
arrow/.style={thick,->},
darrow/.style={thick,<->},
]
\node[io] (n1) at (1,7) {Archivo A, Archivo B};
\node[module] (n2) at (7,7) {Fragmentacion en secuencias y tokenizacion};
\node[module] (n3) at (7,5) {Calculo del maximo emparejamiento de fragmentos};
\node[module] (n4) at (7,3) {Calculo de la subsecuencia de comun mas larga};
\node[module] (n5) at (7,1) {Calculo del indice de similitud};
\node[io] (n6) at (1,1) {Indice de similitud};

\node[algorithm] (n7) at (13,7) {Lexer};
\node[algorithm] (n8) at (13,5) {LCS};
\node[algorithm] (n9) at (13,3) {Hopcroft-Karp};
\node[algorithm] (n10) at (13,1) {Sorencen-Dice};
\node[border] (n11) at (13,7) {};
\node[border] (n12) at (13,5) {};
\node[border] (n13) at (13,3) {};
\node[border] (n14) at (13,1) {};

\draw [arrow] (n1) to (n2);
\draw [arrow] (n2) to (n3);
\draw [arrow] (n3) to (n4);
\draw [arrow] (n4) to (n5);
\draw [arrow] (n5) to (n6);

\draw [darrow] (n7) to (n2);
\draw [darrow] (n8) to (n3);
\draw [darrow] (n9) to (n4);
\draw [darrow] (n10) to (n5);

%\draw (n7.north west) -- ++ (-13.3,7) |- (n7.south west) ;
%\draw (n7.north east) -- ++ (13.3,7) |- (n7.south east) ;

\end{tikzpicture}
\caption{Diagrama de funcionamiento del algoritmo SCBM}
Fuente: Elaboración propia.
\label{diseñoAlg}
\end{figure}
