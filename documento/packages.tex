\usepackage[utf8]{inputenc}
\usepackage{enumerate}
\usepackage[spanish]{babel}
\usepackage{amsmath}
\usepackage{amsfonts}
\usepackage{amssymb}
\usepackage{graphicx}
\usepackage[left=2.6cm,right=2.6cm,top=2.6cm,bottom=2.6cm]{geometry}

% header and footer -< styles
\usepackage{fancyhdr}
\pagestyle{fancy}
\renewcommand{\headrulewidth}{0pt}
\lhead{}\chead{}\rhead{}
\lfoot{}\cfoot{\thepage}\rfoot{}

% enlaces
\usepackage{hyperref}
% sangria
\setlength{\parindent}{1.5cm}
% interlineado
\usepackage{setspace}
\spacing{1.4}
% titulo tabla de contenido
%\renewcommand{\contentsname}{Índice}
% titulo de la bibliografia
%\renewcommand{\refname}{BIBLIOGRAFÍA}
% mostrar bibliografia en contenidos
%\usepackage[nottoc,notlof,notlot]{tocbibind} % Put the bibliography in the ToC

%titulos
\usepackage{titlesec}
\titleformat{\chapter}{\Large\bf\uppercase}{CAPÍTULO \thechapter:}{0.2cm}{}
\titlespacing*{\chapter}{0pt}{-30pt}{10pt}

%pseudocodigo estilo cormen
\usepackage{clrscode3e}
\usepackage{mathrsfs}

%codigo python
\usepackage{listings}
\usepackage{xcolor}
%\usepackage{pxfonts}

% Define a custom color
%\definecolor{blue}{rgb}{0.0, 0.0, 1.0}
%\definecolor{green}{rgb}{0.0, 0.5, 0.0}
% Define a custom style
\lstdefinestyle{myStyle}{
    backgroundcolor=\color{white},
    commentstyle=\bfseries\color{gray},
    basicstyle=\ttfamily,
    breakatwhitespace=false,
    breaklines=true,
    keepspaces=true,
    keywordstyle=\bfseries\color{black},
    identifierstyle=\color{black},
    numbers=left,
    numbersep=6pt,
    showspaces=false,
    showstringspaces=false,
    showtabs=false,
    tabsize=2,
    frame=single,
    lineskip=1pt
}
% Use \lstset to make myStyle the global default
\lstset{style=myStyle}
\renewcommand{\ttdefault}{pcr}
\renewcommand{\lstlistingname}{Programa}% Listing -> Algorithm
\renewcommand{\lstlistlistingname}{Índice de \MakeLowercase{\lstlistingname}s}% List of Listings -> List of Algorithms


%% citas en parentesis
\usepackage[natbibapa]{apacite}

%%  MULTIPLES IMÁGENES JUNTAS
\usepackage{subfig}
% forzar posicion de tablas
\usepackage{float}

% graphs package
\usepackage{tikz}
\usetikzlibrary{arrows}
\usepackage{etoolbox}
\AtBeginEnvironment{tikzpicture}{\shorthandoff{>}\shorthandoff{<}}{}{}

% adjacency list
\usetikzlibrary{matrix}

%flowchart
\usetikzlibrary{shapes.geometric, arrows}

% Token classification
\usetikzlibrary{trees}

%teoremas
\newtheorem{theorem}{Teorema}[section]
